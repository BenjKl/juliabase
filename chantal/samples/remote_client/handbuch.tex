\documentclass[german,11pt]{article}

\usepackage{babel}
\usepackage[utf8]{inputenc}
\usepackage[T1]{fontenc}
\usepackage{mathpple}
\usepackage[a4paper]{geometry}
\usepackage{courier}\usepackage[scaled=0.9]{berasans}
% \usepackage[scaled=0.9]{beramono}
\usepackage{listings}
\usepackage{xspace}
\usepackage{color}

\newcommand{\IEF}{\mbox{IEF-5}\xspace}

\newbox\toll
\setbox\toll=\hbox{\ttfamily\small t}

\lstset{language=python, basicstyle=\ttfamily\small,
  keywordstyle=\small\sffamily\bfseries,
  numberstyle=\footnotesize, basewidth=\wd\toll,
  morekeywords=as,emph={login,logout,new_samples,SixChamberDeposition,SixChamberLayer,SixChamberChannel},
  emphstyle=\bfseries, backgroundcolor=\color[gray]{0.9}, showstringspaces=false,
  escapeinside={(*}{*)}}


\title{Der Chantal Remote-Client}
\author{Torsten Bronger}
\date{24.~September 2008}

\begin{document}
\nonfrenchspacing
\maketitle

\begin{abstract}
  Der „Chantal Remote-Client“ ist ein kleines Programm, das es ermöglicht, von
  einem beliebigen Computer des Instituts Daten in die Datenbank automatisiert
  einzutragen (und auszulesen).

  Warum das ganze?  Damit man an Apparaturen, an denen das sinnvoll ist, die
  Daten, die dort produziert werden und die für die Datenbank interessant sind,
  in dieselbe hineinschreiben kann, ohne daß der Operateur noch mal im Browser
  die ganzen Daten abtippen muß.  Denn letzteres ist umständlich und
  fehleranfällig.
\end{abstract}

\tableofcontents

\section{Das Problem}

Die Datenbank ist nur sinnvoll, wenn da auch alle relevanten Daten drinstehen.
Die beste Möglichkeit, die Datenbank fehlerfrei und vollständig zu halten ist,
daß die datengenerierenden System (Depositionen, Meßapparaturen) automatisch
die Daten an die Datenbank übermitteln.

Nur wie?  Der Browser schickt Daten an die Datenbank über das HTTP-Protokoll.
Und das ist auch das einzige Tor zur Datenbank, damit alle Daten an denselben
Überprüfungen und Authentifizierungen vorbei müssen.  Also müssen die
(Meß"~)""Systeme ihre Daten ebenso über HTTP an den Datenbank-Server schicken.

Das führt zu zwei Problemen:
\begin{enumerate}
\item Die Programmierer im Institut wissen nicht, wie sie Daten per HTTP
  verschicken.  (Ist ja auch keine besonders aufregende Erfahrung~…)
\item Nicht alle Programmiersprachen und Systeme im \IEF \emph{können}
  überhaupt Daten per HTTP verschicken.
\end{enumerate}

Im Institut kommen sehr verschiedene Programmiersysteme zum Einsatz: HT\,Basic,
LabVIEW, Delphi, Python etc.  Außerdem gibt es Systeme, zu denen wir nicht den
Quellcode haben, die aber u.\,U. eine eingebaute Makro- oder Skriptsprache
haben.  Die müssen nun alle in die Lage versetzt werden, in \emph{einfacher}
Form Daten an den Datenbank-Server zu schicken, und eventuell sogar Daten aus
der Datenbank auszulesen.

\section{Die Lösung}

Alle Systeme und alle Programmierer im \IEF sollten zumindest zwei Sachen
können:
\begin{enumerate}
\item Daten in eine Textdatei schreiben bzw.\ aus ihr lesen.
\item Ein externes, rein kommandozeilenbasiertes Programm ausführen.
\end{enumerate}
Und mehr ist für den Chantal Remote-Client nicht nötig!

Will man also aus einem Programm heraus Daten an die Datenbank senden, muß auf
dem jeweiligen Rechner zunächst der Remote-Client installiert sein.  Dann
schreibt man seine „Wunschliste“ in eine Textdatei und ruft ein bestimmtes
externes Programm auf.  Anschließend kann man in einer Log-Datei sehen, ob
alles glattgegangen ist.  Die Log-Datei sollte man natürlich auch durch das
eigene Programm auslesen lassen.

\section{Installation des Remote-Clients}

To be done~…

\section{Grundlegendes}

Technisch gesehen ist die Textdatei, die man schreiben muß, ein
Python-Programm.  Python ist eine höhere Programmiersprache.  Das, was man zum
übermitteln benötigt, ist zwar so einfach, daß man kaum merkt, daß man
irgendwas programmiert, aber theoretisch steht einem der komplette Sprachschatz
von Python zur Verfügung.

Eine Datei für den Remote-Client beginnt immer mit
\begin{lstlisting}
from chantal_remote import *
\end{lstlisting}
Das bedeutet, daß alle Funktionen des Moduls \lstinline{chantal_remote}
eingebunden werden.

In Python ist die Einrückung der Zeilen wichtig.  Achtet deshalb darauf, daß
alle neuen Befehle in der ersten Spalte beginnen.  Solltet ihr Sonderzeichen
(deutsche Umlaute o.\,ä.)\ brauchen, kontaktiert mich.

Alle Zeilen, die mit einer Raute~\lstinline{#} beginnen, werden als
Kommentarzeilen ignoriert:
\begin{lstlisting}
# Programm zum uebersenden von 6-Kammer-Depositionen
from chantal_remote import *
\end{lstlisting}


\subsection{Einloggen und ausloggen}

Es ist wichtig, sich beim Server zu identifizieren.  Dazu muß man sich zu
Beginn der Datei einloggen und sich zuletzt auch wieder ausloggen:
\begin{lstlisting}
from chantal_remote import *

login("r.meier", "mammaistdiebeste")

(*\emph{Hier nun die ganzen Daten~…} *)

logout()
\end{lstlisting}
Dabei ist \lstinline{"r.meier"} der Loginname und
\lstinline{"mammaistdiebeste"} das Paßwort.  Beides ist dasselbe wie für die
freigegebenen Verzeichnissen und Drucker im Institut.  Im folgenden geht es nun
darum, wie man die Daten selber angibt.

Ein Wort zu Paßworten.  Sie sollte man nicht leichtfertig in Dateien
schreiben.  Wenn man es lokal auf seinem Rechner tut und die Datei danach
wieder löscht, ist das okay.  An einer Meßapparatur, zu der niemand sonst
Zugriff hat, mag das auch okay sein.  Man muß aber, wenn man die Apparaur an
jemanden anderes übergibt, sein Paßwort löschen.

Besser ist es jedoch, daß das Meßprogramm verlangt, daß man ein Paßwort
eingibt, und die Textdatei, in der das Paßwort dann steht, sofort nach Gebrauch
wieder löscht.  Noch sicherer ist es, gar nicht erst eine Textdatei zu
schreiben, sondern den Text direkt dem Python-Interpreter zu übergeben.  Das
geht aber u.\,U. nicht mit jeder Programmiersprache.

Außerdem besteht die Möglichkeit, Sammel-Benutzer anzulegen, die mit einem
schwachen Paßwort, das einer begrenzten Zahl von Leuten bekannt ist,
ausgestattet sind.  Zum Beispiel könnte eine Datei so anfangen:
\begin{lstlisting}
from chantal_remote import *
login("pds_benutzer", "rotesocke")
\end{lstlisting}


\subsection{Zeitstempel}

An vielen Stellen muß oder kann man sogenannte Zeitstempel angeben.  Das sind
Strings, die angeben, wann ein Prozeß durchgeführt wurde.  Sowas wird
grundsätzlich immer in der Form
\begin{lstlisting}
"JJJJ-MM-TT HH:MM:SS"
\end{lstlisting}
angegeben, also z.\,B. \lstinline{"2008-08-23 13:34:00"}.

\section{Referenz}

Was folgt, ist eine Auflistung aller bislang unterstützter Operationen anhand
von Beispielen.

\subsection{Proben hinzufügen}

Man kann bis zu 100~Proben auf einen Schlag hinzufügen.  Der folgende Befehl
legt 10 neue Proben an und gibt an, daß sie zur Zeit im MAIKE-Labor liegen:
\begin{lstlisting}
from chantal_remote import *
login("r.meier", "mammaistdiebeste")
new_samples(10, "MAIKE-Labor")
logout()
\end{lstlisting}
Es gibt noch einige optionale Parameter, nämlich Substrat, Zeitstempel, Zweck,
Tags und Gruppe, bzw.\ \lstinline{substrate}, \lstinline{timestamp},
\lstinline{purpose}, \lstinline{tags} und \lstinline{group}.  Man könnte also
auch schreiben:
\begin{lstlisting}
new_samples(10, "MAIKE-Labor",
            substrate="asahi-u",
            timestamp="2008-08-23 13:34:00",
            group="SiC")
\end{lstlisting}
Aber \lstinline{"asahi-u"} ist eh Default, und wenn kein Zeitstempel angegeben
wird, gilt \emph{jetzt}.  Bei optionalen Parametern schreibt man besser den
Parameternamen mit Gleichheitszeichen davor.


\subsection{6-Kammer-Depositionen}

\begin{lstlisting}[numbers=left]
samples = new_samples(5, "6-Kammer-Labor")  (*\label{sc:Proben}*)

deposition = SixChamberDeposition(samples)  (*\label{sc:deposition}*)
deposition.timestamp = "2008-09-15 22:29:00"    (*\label{sc:attribute}*)

layer = SixChamberLayer(deposition)   (*\label{sc:layer}*)
layer.chamber = "#1"                              (*\label{sc:chamber}*)

channel1 = SixChamberChannel(layer)               (*\label{sc:channel}*)
channel1.number = 1
channel1.gas = "SiH4"
channel1.flow_rate = 1

channel2 = SixChamberChannel(layer)               (*\label{sc:another-channel}*)
channel2.number = 2
channel2.gas = "SiH4"
channel2.flow_rate = 2

deposition.submit()                   (*\label{sc:submit}*)
\end{lstlisting}

Okay, was passiert hier?\bigskip

\begin{itemize}
\item In Zeile~\ref{sc:Proben} hole ich mir 5~neue Proben.  Da ich mit den
  Proben noch was vorhabe, speichere ich sie in der Variablen
  \lstinline|samples|.
\item Dann erzeuge ich eine neue 6-Kammer-Deposition in
  Zeile~\ref{sc:deposition} mit \lstinline|SixChamberDeposition|, übergebe ihr
  die Proben als Parameter und speichere sie in einer Variablen.
\item In Zeile~\ref{sc:attribute} kann man sehen, wie man einem Objekt
  Eigenschaften zuweist: Nämlich mit der beliebten Punkt-Notation.  Mit
  \lstinline|.timestamp = |…\ kann man beispielsweise der Deposition einen
  Zeitstempel aufdrücken.
\item Eine Deposition braucht mindestens eine Schicht.  Die legen wir in
  Zeile~\ref{sc:layer} an.  Man muß ihr die Depositions-Variable übergeben, zu
  der sie gehört.  In Zeile~\ref{sc:chamber} weisen wir ihr eine Kammer zu.
\item In Zeile~\ref{sc:channel} legen wir den ersten Channel an und verbinden
  ihn mit der Schicht, analog zu vorhin.  Außerdem weisen wir dem neuen Channel
  ein paar Attribute zu.
\item Damit's nicht zu langweilig ist, kommt in Zeile~\ref{sc:another-channel}
  noch ein Channel zu derselben Schicht hinzu.
\item Schließlich wird in Zeile~\ref{sc:submit} die Deposition mit allen
  Schichten und Channels in die Datenbank übertragen. 
\end{itemize}

\end{document}
